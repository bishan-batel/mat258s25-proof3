\documentclass{exam}

\usepackage{amsfonts}
\usepackage{amssymb}
\usepackage{mathtools}
\usepackage{braket}
\usepackage{forloop}
\usepackage{amsthm}
\usepackage[backend=biber]{biblatex}

\theoremstyle{plain}
\newtheorem{assumption}{Assumption}

\theoremstyle{definition}
\newtheorem{definition}{Definition}

\newtheorem{lemma}{Lemma}


% huge align
\newcommand{\ha}[1]{{\huge{\begin{align*}#1\end{align*}}}}



% P & S Are excluded 
% \newcommand{\A}[0]{{\mathbb A}}
% \newcommand{\B}[0]{{\mathbb B}}
% \newcommand{\C}[0]{{\mathbb C}}
% \newcommand{\D}[0]{{\mathbb D}} 
% \newcommand{\E}[0]{{\mathbb E}} 
% \newcommand{\F}[0]{{\mathbb F}} 
% \newcommand{\G}[0]{{\mathbb G}} 
% \newcommand{\H}[0]{{\mathbb H}} 
% \newcommand{\I}[0]{{\mathbb I}} 
% \newcommand{\J}[0]{{\mathbb J}} 
% \newcommand{\K}[0]{{\mathbb K}} 
% \newcommand{\L}[0]{{\mathbb L}} 
% \newcommand{\M}[0]{{\mathbb M}} 
\newcommand{\N}[0]{{\mathbb N}}
% \newcommand{\O}[0]{{\mathbb O}} 
% \newcommand{\P}[0]{{\mathbb P}} 
\newcommand{\Q}[0]{{\mathbb Q}}
\newcommand{\R}[0]{{\mathbb R}}
% \renewcommand{\S}[0]{{\mathbb S}}
% \newcommand{\T}[0]{{\mathbb T}}
% \newcommand{\U}[0]{{\mathbb U}}
% \newcommand{\V}[0]{{\mathbb V}}
% \newcommand{\W}[0]{{\mathbb W}}
% \newcommand{\X}[0]{{\mathbb X}}
% \newcommand{\Y}[0]{{\mathbb Y}}
\newcommand{\Z}[0]{{\mathbb Z}}

% Calculus
% \newcommand{\d}[0]{{\mathrm{d}}}
\newcommand{\deriv}[2]{ \frac{ \d{#1} }{ \d{#2} } }
\newcommand{\pderiv}[2]{ \frac{ \partial{#1} }{ \partial{#2} } }

\newcommand{\nderiv}[3]{ \frac{ \d^{#1}{#2} }{ \d{#3}^{#1} } }
\newcommand{\npderiv}[3]{ \frac{ \partial^{#1}{#2} }{ \partial{#3}^{#1} } }

% Linear Algebra 
\renewcommand{\vector}[1]{ \overrightarrow{#1} }
\newcommand{\vecn}[1]{ {\hat #1} }

\newcommand{\mat}[1]{{ \begin{bmatrix} #1 \end{bmatrix} }}
\newcommand{\mats}[1]{{ \ba{\begin{smallmatrix} #1 \end{smallmatrix}} }}

\newcommand{\pmat}[1]{{ \begin{pmatrix} #1 \end{pmatrix} }}
\newcommand{\pmats}[1]{{ \pa{\begin{smallmatrix} #1 \end{smallmatrix}} }}

\newcommand{\emat}[1]{{ \begin{ematrix} #1 \end{ematrix} }}
\newcommand{\emats}[1]{{ \begin{smallmatrix} #1 \end{smallmatrix} }}

\newcommand{\vmat}[1]{{ \begin{vmatrix} #1 \end{vmatrix} }}

\newcommand{\rowechelon}[1]{{
			\left[\begin{array}{ccc|c} #1 \end{array}\right]
		}}

\newcommand{\augmented}[2]{{
			\left[\begin{array}{#1} #2 \end{array}\right]
		}}

% Generic Notatino
\newcommand{\paren}[1]{{ \left(#1\right) }}

\newcommand{\pa}[1]{{ \left(#1\right) }}
\newcommand{\ba}[1]{{ \left[#1\right] }}


\newcommand{\llet}[0]{ {\text{let } } }
\newcommand{\undefined}[0]{ {\text{undefined.} } }

\newcommand{\op}[1]{ {\operatorname{#1} } }

\newcommand{\brt}[2]{ {\root {#1} \of {#2} } }

\newcommand{\proj}[1]{ { \op{proj}_{#1} }}
\newcommand{\projperp}[1]{ { \op{proj}_{#1\perp} } }

\newcommand{\norm}[1]{{ {\left\lVert #1 \right\rVert} }}
\newcommand{\norms}[1]{{ {\lVert #1 \rVert} }}

% CS 
\newcommand{\hex}[1]{{ \pa{\mathrm{#1}}_{16} }}
\newcommand{\bin}[1]{{ \pa{#1}_{2} }}
\newcommand{\binb}[2]{{ \pa{#1}^{#2}_{2} }}
\newcommand{\dec}[1]{{ \pa{#1}_{10} }}


\newcommand{\true}[0]{{ \mathrm{true} }}
\newcommand{\false}[0]{{ \mathrm{false} }}

\renewcommand{\ba}[1]{{ \left[ {#1} \right] }}

\newcommand{\ceil}[1]{{ \left\lceil {#1} \right\rceil }}
\newcommand{\floor}[1]{{ \left\lfloor {#1} \right\rfloor }}

\newcommand{\ang}[1]{{ \left\langle {#1} \right\rangle }}

\newcommand{\transpose}[1]{ { {#1}^{\intercal} } }





\addbibresource{references.bib}

\begin{document}

\title{MAT258S25 Proof 5}
\author{Kishan S Patel}
\maketitle

\renewcommand{\qedsymbol}{QED}


\begin{lemma}[$\mathcal L_1$]
	The $n$th triangular number can be calculated by the explicit formula:
	$$
		T_n = \sum_{i=1}^{n} i
	$$
\end{lemma}

\begin{lemma}[$\mathcal L_2$]
	If some number $m^2$ is both odd and a perfect square, $m$ must be odd.
\end{lemma}

\begin{lemma}[$\mathcal L_3$]
	For any odd positive integer $m$, $\exists n \in \Z, n\ge 0 : m=2n+1$.
\end{lemma}

\begin{lemma}[$\mathcal L_4$]
	$\Z^{+}$ is closed under addition, subtraction, and multiplication.
\end{lemma}

Let $T_m=1+2+3+\cdots+m$ where $m \in \Z$  with $m \ge 1$ ($T_m$ is the $m$th triangular number).


\begin{questions}

	\begin{question}
	Use induction to prove $\forall m \in (\Z^{+} \setminus \set{0} ) : T_m = \frac{m(m+1)}{2} $.
	\begin{proof}

		Let $P_1(m)$ be the proposition that $T_m = \frac{m(m+1)}{2}$ is true.

		\begin{align*}
			\mathcal L_1 \implies			T_1 & = \sum_{i=1}^{1}i = 1                       \\
			P_1(1) \implies T_1         & = \frac{m(m+1)}{2}                          \\
			                            & = \frac{1(1+1)}{2}                          \\
			                            & = 1                                         \\
			                            & \therefore\,         P_1(1) \text{is true.}
		\end{align*}
		\begin{align*}
			\mathcal L_1 \implies			T_{k+1}  & = \sum_{i=1}^{k+1}i       \\
			                                 & = \sum_{i=1}^{k}i + (k+1) \\
			\therefore															T_{k+1} & = T_k + (k+1)
		\end{align*}
		\begin{align*}
			P_1(k)   & \implies T_k                             = \frac{k(k+1)}{2} \\
			P_1(k+1) & \implies 	T_{k+1}  =\frac{(k+1)(k+2)}{2}
		\end{align*}
		\begin{align*}
			P_1(k) \wedge \mathcal{L}_1 \implies T_{k+1} & = \frac{k(k+1)}{2} + (k+1)   \\
			                                             & = \frac{k(k+1) + 2(k+1)}{2}  \\
			                                             & = \frac{k^2 + k + 2k + 2}{2} \\
			                                             & = \frac{k^2 + 3k + 2}{2}     \\
			                                             & =\frac{(k+1)(k+2)}{2}        \\
			\therefore P_1(k)\wedge \mathcal{L}_1        & \implies P_1(k+1)
		\end{align*}
		\begin{align*}
			\mathcal{L}_1                       & \implies P_1(1)             \\
			\mathcal{L}_1 \wedge		P_1(k)        & \implies P_1(k+1)           \\
			\therefore \forall m          \in T & : P_1(m)  \text{ is true. }
		\end{align*}


	\end{proof}
\end{question}

	\pagebreak
\begin{question}

	Prove that the positive integer $n$ is a triangular number ($P_2$) if and only if $8n+1$ is a perfect square ($P_3$).

	\begin{proof}

		\begin{align*}
			P_3 \implies			m \in \Z^{+} & : m^2 = 8n+1
		\end{align*}
		\begin{align*}
			\llet a       & = 4n                            \\
			8n+1          & = 2a+1                          \\
			\mathcal{L}_3 & \implies {8n+1} \text{ is odd.} \\
			\mathcal{L}_2 & \implies {m} \text{ is odd.}
		\end{align*}
		\begin{align*}
			m^2                     & = 8n + 1            \\
			\llet b \in \Z, b \ge 0 & : m = 2b + 1;       \\
			(2b+1)^2                & = 8n+1              \\
			4b^2+4b+1               & = 8n+1              \\
			n                       & = \frac{4b^2+4b}{8} \\
			n                       & = \frac{b^2+b}{2}   \\
			n                       & = \frac{b(b+1)}{2}  \\
			n               = T_b   & \implies P_2        \\
			\therefore P_3          & \implies P_ 2
		\end{align*}
		\begin{align*}
			P_2 \implies \exists k \in \Z^{+}  : n & = \frac{k(k+1)}{2}      \\
			8n+1                                   & = 8\frac{k(k+1)}{2} + 1 \\
			                                       & = 4k(k+1) + 1           \\
			                                       & = 4k^2 + 4k +1          \\
			                                       & = 4\pa{k+\frac{1}{2}}^2 \\
			                                       & = \pa{16k+8}^2          \\
			\llet m                                & = 16k + 8               \\
			\mathcal{L}_4 \wedge k \in \Z^{+}      & \implies m \in Z^+      \\
			8n + 1                                 & = m^2                   \\
			\therefore			P_2                       & \implies P_3
		\end{align*}

		$$ \ba{(P_2 \implies P_3) \wedge (P2 \impliedby P_3)} \implies P_2 \iff P_3 $$


	\end{proof}

\end{question}


\end{questions}

\end{document}

\begin{question}
	Use induction to prove $\forall m \in (\Z^{+} \setminus \set{0} ) : T_m = \frac{m(m+1)}{2} $.
	\begin{proof}

		Let $P_1(m)$ be the proposition that $T_m = \frac{m(m+1)}{2}$ is true.

		\begin{align*}
			\mathcal L_1 \implies			T_1 & = \sum_{i=1}^{1}i = 1                       \\
			P_1(1) \implies T_1         & = \frac{m(m+1)}{2}                          \\
			                            & = \frac{1(1+1)}{2}                          \\
			                            & = 1                                         \\
			                            & \therefore\,         P_1(1) \text{is true.}
		\end{align*}
		\begin{align*}
			\mathcal L_1 \implies			T_{k+1}  & = \sum_{i=1}^{k+1}i       \\
			                                 & = \sum_{i=1}^{k}i + (k+1) \\
			\therefore															T_{k+1} & = T_k + (k+1)
		\end{align*}
		\begin{align*}
			P_1(k)   & \implies T_k                             = \frac{k(k+1)}{2} \\
			P_1(k+1) & \implies 	T_{k+1}  =\frac{(k+1)(k+2)}{2}
		\end{align*}
		\begin{align*}
			P_1(k) \wedge \mathcal{L}_1 \implies T_{k+1} & = \frac{k(k+1)}{2} + (k+1)   \\
			                                             & = \frac{k(k+1) + 2(k+1)}{2}  \\
			                                             & = \frac{k^2 + k + 2k + 2}{2} \\
			                                             & = \frac{k^2 + 3k + 2}{2}     \\
			                                             & =\frac{(k+1)(k+2)}{2}        \\
			\therefore P_1(k)\wedge \mathcal{L}_1        & \implies P_1(k+1)
		\end{align*}
		\begin{align*}
			\mathcal{L}_1                                               & \implies P_1(1)             \\
			\mathcal{L}_1 \wedge		P_1(k)                                & \implies P_1(k+1)           \\
			\therefore \forall m          \in \op{dom}({\mathcal{L}_1}) & : P_1(m)  \text{ is true. }
		\end{align*}


	\end{proof}
\end{question}
